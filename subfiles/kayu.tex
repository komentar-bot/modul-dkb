%! TEX root = kayu.tex
% the main TeX file which is intended to compile, :VimtexReload after adjustment
   % this file should be included in the main file
% :h vimtex-tex-root
\documentclass[../dkb.tex]{subfiles}
%ensure the class of docs

\begin{document}

\section{Kayu}\label{sec:Kayu} % (fold)

% section Kayu (end)

Kayu dan bambu adalah dua material alami yang sering digunakan dalam berbagai aplikasi konstruksi dan kerajinan. Kayu berasal dari pohon dan dibedakan menjadi dua kategori utama: kayu keras, yang berasal dari pohon berdaun lebar seperti jati dan mahoni, dan kayu lunak, yang berasal dari pohon konifer seperti pinus dan cemara \citep{astuti2019}. Di sisi lain, bambu adalah tanaman berbatang keras yang termasuk dalam keluarga rumput-rumputan, dengan jenis-jenis yang terkenal seperti bambu petung Dendrocalamus asper dan bambu hitam Phyllostachys nigra, yang dikenal karena pertumbuhannya yang cepat dan kekuatannya yang tinggi (Sutrisno, A., \& Prabowo, 2020). Keduanya memiliki karakteristik unik yang menjadikannya pilihan populer dalam konstruksi dan desain berkelanjutan. Bambu dianggap lebih ramah lingkungan dibandingkan kayu karena dapat tumbuh kembali dengan cepat setelah dipanen.

Bambu dan kayu masing-masing memiliki kelebihan dan kekurangan yang signifikan dalam penggunaannya. Kelebihan bambu terletak pada pertumbuhannya yang sangat cepat dan kemampuannya untuk tumbuh kembali setelah dipanen, menjadikannya sumber daya yang lebih berkelanjutan dibandingkan kayu (Kurniawan, H. P., \& Wibowo, 2021). Selain itu, bambu memiliki kekuatan tarik yang tinggi dan fleksibilitas yang baik, sehingga sering digunakan dalam konstruksi dan pembuatan furnitur. Namun, bambu juga memiliki kekurangan, seperti ketahanannya yang lebih rendah terhadap kelembapan dan serangan hama jika tidak dirawat dengan baik (Fitria, R. A., \& Utami, 2018). Di sisi lain, kayu memiliki daya tahan yang baik dan estetika yang menarik, tetapi proses pembudidayaan pohon kayu memerlukan waktu yang lebih lama, dan penebangan yang tidak terkelola dapat menyebabkan deforestasi. Oleh karena itu, pemilihan antara bambu dan kayu harus mempertimbangkan konteks penggunaan dan dampak lingkungan.

\subsection{Bambu dan Kayu sebagai Bahan Konstruksi dan Non Konstruksi}%
\label{sub:Bambu dan Kayu sebagai Bahan Konstruksi dan Non Konstruksi}


Bambu dan kayu memiliki karakteristik yang membuat keduanya sangat berharga sebagai bahan konstruksi dan nonkonstruksi. Bambu, yang merupakan tanaman berbatang keras dari keluarga rumput-rumputan, memiliki kekuatan yang tinggi, fleksibilitas, dan kemampuan untuk tumbuh dengan cepat, sehingga menjadikannya pilihan yang ramah lingkungan untuk berbagai aplikasi, termasuk sebagai bahan bangunan, furnitur, dan kerajinan tangan (Tjenggoro \& Prasetyo, 2018). Di sisi lain, kayu, yang berasal dari pohon, dikenal karena daya tahannya, estetika yang menarik, dan kemudahan dalam pengolahan, sehingga sering digunakan dalam konstruksi bangunan, pembuatan furniture, dan berbagai produk kerajinan. Meskipun keduanya memiliki keunggulan, bambu lebih unggul dalam hal keberlanjutan karena dapat tumbuh kembali dengan cepat setelah dipanen, sementara kayu memerlukan waktu yang lebih lama untuk tumbuh dan dapat menyebabkan deforestasi jika tidak dikelola dengan baik.

Bambu merupakan material yang tepat digunakan pada bangunan – bangunan di Indonesia yang memiliki udara yang berangin dan rawan bencana gempa. Bambu memiliki kemampuan gaya tekan dan gaya lentur yang dapat menyesuaikan kondisi alam di Indonesia (Widyowijatnoko Aditra, 2018)) Selain itu, bambu merupakan material sustainable yang penggunaannya dapat menggantikan kayu. Material bambu dalam konstruksi bangunan memiliki potensi kekuatan, kekakuan dan keamanan yang dapat disejajarkan dengan baja, beton dan kayu. Pemasangan bambu sebagai bahan kontruksi memiliki teknik sambungan pada bambu dibedakan menjadi 2 jenis yaitu teknik tradisional dan teknik modern. Teknik tradisional biasanya membutuhkan penggunaan tali bambu. Meskipun dapat dikerjakan oleh kemampuan kerja manusia, namun simpul menggunakan tali bambu tidak efisien secara struktural dan kurangnya kekuatan pada sambungan antar batang bambu. Semakin berkembangnya jaman, metode sambungan pada bambu menjadi lebih sederhana menggunakan mur dan baut.

\subsubsection{Sistem Sambungan Tradisional}%
\label{ssub:Sistem Sambungan Tradisional}

\begin{enumerate}
	\item Sambungan Pemanjang
	\item[] Sambungan pemanjang berfungsi untuk memperpanjang atau menghubungkan dua batang bambu dengan cara menyambungkan 2 buah bambu.

	      \begin{figure}
		      \begin{center}
			      \includegraphics[width=.82\textwidth]{../figures/sambpemanjang.jpeg}
			      % filename may not contain space but _
		      \end{center}
		      \caption{Sambungan Pemanjang}
		      \label{fig:sambpemanjang}
	      \end{figure}

	\item Sambungan Ortogonal

	      Sambungan ortogonal adalah metode penyambungan dua batang bambu yang memiliki orientasi berbeda, baik secara horizontal maupun vertikal, atau dengan sudut diagonal tertentu.

	      \begin{figure}
		      \begin{center}
			      \includegraphics[width=.82\textwidth]{../figures/sambortogonal.jpeg}
			      % filename may not contain space but _
		      \end{center}
		      \caption{Sambungan Ortogonal}
		      \label{fig:samborto}
	      \end{figure}

	\item Sambungan Sudut
	\item[] Sambungan sudut adalah metode penyambungan dua atau lebih batang bambu di bagian ujungnya, sehingga membentuk kemiringan atau sudut tertentu.

	      \begin{figure}
		      \begin{center}
			      \includegraphics[width=.82\textwidth]{../figures/sambsudut.jpeg}
			      % filename may not contain space but _
		      \end{center}
		      \caption{Sambungan Sudut}
		      \label{fig:sambsudut}
	      \end{figure}

	\item Sambungan Menerus
	\item[] Sambungan menerus adalah metode penyambungan dua batang bambu dengan orientasi yang berbeda, baik horizontal maupun vertikal, di mana salah satu bambu dilubangi dan bambu lainnya disambungkan melalui lubang tersebut.

	      \begin{figure}
		      \begin{center}
			      \includegraphics[width=.82\textwidth]{../figures/sambmenerus.jpeg}
			      % filename may not contain space but _
		      \end{center}
		      \caption{Sambungan Menerus}
		      \label{fig:sambmenerus}
	      \end{figure}
\end{enumerate}

\subsubsection{Sistem Sambungan Modern}%
\label{ssub:Sistem Sambungan Modern}

\begin{enumerate}
	\item Sambungan Pelat Kayu
	\item[] Sistem sambungan yang melibatkan lebih dari dua batang bambu menggunakan plat kayu sebagai penghubung, yang kemudian direkatkan dengan menggunakan mur dan baut.

	      \begin{figure}
		      \begin{center}
			      \includegraphics[width=.82\textwidth]{../figures/sambplat.jpeg}
			      % filename may not contain space but _
		      \end{center}
		      \caption{Sambungan Plat Kayu}
		      \label{fig:sambplat}
	      \end{figure}

	\item Sambungan Baut
	\item[] Sambungan baut adalah metode penyambungan dua batang bambu atau lebih yang diatur dan disatukan menggunakan mur dan baut sebagai penghubung.

	      \begin{figure}
		      \begin{center}
			      \includegraphics[width=.82\textwidth]{../figures/samb_baut.jpeg}
			      % filename may not contain space but _
		      \end{center}
		      \caption{Sambungan Baut}
		      \label{fig:sambbaut}
	      \end{figure}


	      Struktur kayu adalah suatu bentuk konstruksi yang terdiri dari elemen-elemen yang terbuat dari kayu. Dalam perkembangannya, penggunaan struktur kayu telah menjadi alternatif yang populer dalam perencanaan proyek-proyek sipil, termasuk di antaranya adalah rangka kuda-kuda, gelagar jembatan, struktur perancah, kolom, dan balok lantai bangunan. Penting untuk dicatat bahwa perancangan struktur kayu tidak seharusnya hanya didasarkan pada pengalaman atau perkiraan semata. Sebaliknya, perhitungan yang akurat harus dilakukan berdasarkan prinsip-prinsip ilmu gaya. Meskipun demikian, pengalaman dari struktur kayu yang telah ada sebelumnya dapat memberikan wawasan dan panduan awal yang berguna. Oleh karena itu, mulai dari penetapan beban yang bekerja hingga perhitungan gaya-gaya yang terjadi pada struktur, serta penentuan ukuran dan sambungan, harus dilakukan secara rasional dan sesuai dengan norma serta peraturan yang berlaku.

\end{enumerate}

\subsection{Aturan Penetapan Pembebanan}\label{sub:Aturan Penetapan Pembebanan} % (fold)

Penetapan besarnya muatan-muatan (beban) yang bekerja pada struktur, harus mengacu pada ketetapan / peraturan yang berlaku, misalnya : Dana Normalisasi Indonesia NI-02006, NI- 02007, Peraturan-peraturan pembebanan yang dikeluarkan oleh Departemen Pekerjaan Umum, Tenaga Perum Kereta Api, dan sebagainya.

Pembebanan dapat dibagi menjadi tiga kategori berdasarkan jenis dan durasinya. Beban tetap adalah beban yang berlangsung lebih dari 3 bulan, mencakup beban statis yang terus menerus, seperti berat sendiri, tekanan tanah, tekanan air, barang-barang di gudang, dan kendaraan di atas jembatan. Beban sementara, di sisi lain, berlangsung kurang dari 3 bulan dan meliputi muatan bergerak yang bersifat tidak tetap atau terus menerus, seperti berat orang yang berkumpul di ruangan pertemuan atau kantor. Terakhir, beban khusus merupakan kombinasi dari beban tetap atau sementara yang ditambah dengan beban tertentu yang bekerja pada struktur atau bagian struktur. Beban ini dapat disebabkan oleh faktor-faktor seperti perbedaan suhu, pengangkatan atau penurunan, penurunan pondasi, susut, serta gaya-gaya tambahan dari beban hidup seperti gaya rem dari kendaraan, gaya sentrifugal, dan gaya dinamis dari mesin-mesin serta pengaruh khusus lainnya.

Ukuran penampang balok minimum yang digunakan mengacu pada Pasal 9 dan Pasal 10 PKKI (Peraturan Konstrusi Kayu Indonesia, 1961), yang isi pokoknya terdapat pada pernyataan dibawah ini. PKKI Pasal 9:

\begin{enumerate}
	\item Ukuran salah satu sisi (lebar/tinggi) balok kayu yang digunakan sebagai bagian struktur rangka batang paling kecil adalah 4 cm, dengan luas penampangnya lebih besar 32 cm2.
	\item Apabila batang itu terdiri lebih dari satu bagian maka syaratsyarat tersebut untuk keseluruhan tampang.
	\item Untuk struktur dengan paku atau perekat, syarat-syarat tersebut tidak berlaku.
\end{enumerate}


PKKI Pasal 10 :
\item Perhitungan ukuran dan luas penampang akibat adanya perlemahan, pada batang-batang tarik dan bagian-bagian struktur yang dibebani dengan tegangan lentur harus diperhitungkan.
\item Untuk batang yang menahan tegangan desak, perlemahan akibat alat sambung tidak perlu diperhitungkan (dengan catatan bahwa lubang tersebut tertutup oleh alat sambung).
\item Tetapi apabila dalam kenyataannya lubang tersebut tidak tertutup, maka lubang tersebut harus diperhitungkan sebagai perlemahan.

% subsection Aturan Penetapan Pembebanan (end)

\subsection{Lendutan Maksimum yang Diijinkan}\label{sub:Lendutan Maksimum yang Diijinkan} % (fold)

Penetapan besarnya lendutan yang diijinkan pada struktur kayu, diatur melalui Pasal 12 ayat 5 PKKI, dengan isi pokok sebagai berikut:

\begin{enumerate}
	\item Lendutan maksimum yang diperbolehkan, untuk balok pada struktur terlindung < L/300 panjang bentang, dengan L adalah panjang bentang.
	\item Untuk balok pada struktur tidak terlindung < L/400 panjang bentang.
	\item Untuk balok yang digunakan pada struktur kuda-kuda, misalnya gording, < L/200 panjang bentang. 4. Untuk rangka batang yang tidak terlindung < L/700 panjang bentang.
\end{enumerate}

Pada perencanaan perhitungan batang desak dan batang terlentur beberapa rumus membutuhkan Modulus Elastis Kayu (dilambangkan dengan huruf E). Modulus Elastis diperlukan untuk menghitung perubahan bentuk elastis, besarnya berbedabeda menurut kelas kuat kayunya, sebagaimana tersaji pada

\begin{table}[htb]
	\caption{Besaran Modulus Elastis (E) Kayu Sejajar Serat}
	\label{tb:moduluselastis}
	\begin{tabularx}{.8\textwidth}{XX}A
		\textbf{Kelas Kuat Kayu}  &  \textbf{Modulus Elastis E (kg/cm\textsuperscript{2})}  \\
		\hline
		I                         &  125.000  \\
		\hline
		II                        &  100.000  \\
		\hline
		III                       &  80.000  \\
		\hline
		IV                        &  60.000  \\
		\hline
	\end{tabularx}
\end{table}
% subsection Lendutan Maksimum yang Diijinkan (end)

\subsection{Desain Bangunan dengan Material Bambu dan Kayu}\label{sub:Desain Bangunan dengan Material Bambu dan Kayu} % (fold)


Bambu dan kayu merupakan dua material alami yang sering digunakan dalam desain bangunan karena karakteristiknya yang ramah lingkungan, estetik, dan fungsional. Material ramah lingkungan adalah bahan bangunan yang diproduksi, digunakan, dan didaur ulang dengan mempertimbangkan efisiensi energi, pengurangan limbah, serta dampak lingkungan yang rendah. Material ini biasanya bersifat alami, dapat diperbarui, atau memiliki siklus hidup yang panjang. Berikut adalah tabel perbandingan bambu dan kayu sebagai material bangunan.
\begin{table}[htb]
	\caption{Perbandingan Bambu dan Kayu}
	\label{tb:bandingkayu}
	\begin{tabularx}{.83\textwidth}{p{4cm}XX}

		\textbf{Aspek}       &  \textbf{Bambu}                                                                     &  \textbf{Kayu}  \\
		\hline
		Ketersediaan         &  Tumbuh cepat (3-5 tahun) dan melimpah di daerah tropis.                            &  Memerlukan waktu lebih lama untuk tumbuh, terutama kayu keras (10-50 tahun).  \\
		\hline
		Keberlanjutan        &  Sangat ramah lingkungan, dapat diperbarui dengan cepat.                            &  Ramah lingkungan jika berasal dari hutan yang dikelola berkelanjutan (sertifikasi FSC/PEFC).  \\
		\hline
		Kekuatan             &  Rasio kekuatan terhadap berat tinggi; setara dengan baja dalam beberapa aplikasi.  &  Kuat dan tahan lama, terutama kayu keras seperti jati, ulin, atau merbau.  \\
		\hline
		Daya Tahan           &  Rentan terhadap kelembapan, rayap, dan pembusukan tanpa pengawetan khusus.         &  Lebih tahan lama, terutama kayu keras, tetapi tetap membutuhkan perlindungan dari rayap dan jamur.  \\
		\hline
		Estetika             &  Tekstur halus, alami, dan memberikan nuansa tropis yang khas.                      &  Tekstur dan pola alami yang bervariasi, memberikan kesan hangat dan elegan.  \\
		\hline
		Biaya                &  Lebih murah, terutama di daerah dengan pasokan lokal yang melimpah.                &  Relatif lebih mahal, terutama untuk jenis kayu keras berkualitas tinggi.  \\
		\hline
		Penggunaan Struktur  &  Cocok untuk struktur ringan seperti rangka, tiang, dan dinding anyaman.            &  Cocok untuk struktur utama, seperti balok, rangka atap, lantai, dan dinding panel.  \\
		\hline
		Isolasi Termal       &  Memberikan insulasi termal alami, cocok untuk iklim tropis.                        &  Isolasi termal lebih baik dibandingkan dengan bahan logam atau beton.  \\
	\end{tabularx}
\end{table}

% subsection Desain Bangunan dengan Material Bambu dan Kayu (end)

\subsection{Studi Kasus Bangunan Ramah Lingkungan menggunakan
	Bambu dan Kayu}\label{sub:Studi Kasus Bangunan Ramah Lingkungan menggunakan
	Bambu dan Kayu} % (fold)

Studi kasus ini bertujuan untuk menunjukkan penerapan kedua material tersebut dalam desain yang berkelanjutan dan ramah lingkungan. Berikut adalah beberapa contoh yang dapat dijadikan referensi:

\begin{enumerate}
	\item Green Village, Bali (Bangunan Bambu)
	\item[] Green Village adalah sebuah proyek perumahan dan komunitas yang dibangun dengan menggunakan bambu sebagai bahan utama. Terletak di tengah hutan tropis, proyek ini dirancang dengan fokus pada keberlanjutan, dengan memanfaatkan sumber daya lokal secara maksimal. Rumahrumah di Green Village menggunakan bambu untuk struktur utama, dinding, atap, dan elemen interior. Bambu dipilih karena sifatnya yang ringan, kuat, dan mudah didapatkan di daerah sekitar. Atap yang menggunakan bambu juga dilengkapi dengan panel surya untuk memenuhi kebutuhan energi rumah, mendukung keberlanjutan proyek.

	      \begin{figure}
		      \begin{center}
			      \includegraphics[width=.82\textwidth]{../figures/greenvillage.jpeg}
			      % filename may not contain space but _
		      \end{center}
		      \caption{Green Village}
		      \label{fig:greenvillage}
	      \end{figure}

	      Penggunaan bambu dalam proyek ini berhasil menciptakan suasana yang natural dan harmonis dengan lingkungan sekitar, sekaligus menawarkan solusi ramah lingkungan dan ekonomis. Desain bangunan mengutamakan efisiensi energi dan minimalkan dampak terhadap lingkungan. Green Village berkomitmen untuk mengurangi jejak karbon dengan menggunakan bambu, yang merupakan bahan yang cepat tumbuh dan terbarukan.

	\item Rumah Adat Toraja (Kayu Tradisional)
	\item[] Rumah adat Toraja adalah contoh penggunaan kayu dalam konstruksi bangunan tradisional Indonesia. Rumah ini terbuat dari kayu yang diambil dari hutan setempat, dan dibangun dengan teknik yang diwariskan turun-temurun. Struktur rumah didominasi oleh kayu keras, seperti kayu ulin, yang dikenal tahan terhadap cuaca dan serangga. Atap yang curam dan dilapisi dengan bahan alami memberikan insulasi yang baik.

	      \begin{figure}
		      \begin{center}
			      \includegraphics[width=.82\textwidth]{../figures/tongkonan.jpeg}
			      % filename may not contain space but _
		      \end{center}
		      \caption{Tongkonan}
		      \label{fig:tongkonan}
	      \end{figure}

	\item[] Pemanfaatan kayu sebagai bahan bangunan alami mengurangi dampak negatif terhadap lingkungan, dan rumahrumah ini mampu bertahan lama jika dirawat dengan baik. Hutan yang menjadi sumber kayu untuk pembangunan rumah adat Toraja dikelola secara berkelanjutan, dengan memperhatikan keseimbangan ekosistem. Rumah adat Toraja tidak hanya memiliki daya tahan tinggi, tetapi juga memberikan kenyamanan dengan ventilasi alami dan penggunaan bahan lokal.

	\item Eko-Village Cikadang (Bambu dan Kayu)
	\item[] Eko-Village Cikidang adalah sebuah komunitas ramah lingkungan yang menggabungkan bambu dan kayu dalam desain bangunannya. Proyek ini bertujuan untuk menciptakan rumah yang hemat energi, berkelanjutan, dan terintegrasi dengan alam. Berlokasi di Sukabumi, Jawa Barat, Bangunan dirancang untuk memanfaatkan sumber daya lokal dengan efisiensi tinggi, mengurangi jejak karbon dan limbah. Struktur bangunan utama menggunakan bambu untuk tiang dan rangka, sedangkan kayu digunakan untuk dinding dan lantai. Sistem pengolahan air hujan dan penggunaan energi terbarukan (panel surya) menjadi bagian integral dari desain. Menggunakan bambu. dan kayu secara bersamaan memberikan fleksibilitas desain serta memperkuat keberlanjutan proyek secara keseluruhan

\end{enumerate}






% subsection Studi Kasus Bangunan Ramah Lingkungan menggunakan

\subsection{Kesimpulan}\label{sub:Kesimpulan} % (fold)

Secara keseluruhan, bambu dan kayu sebagai bahan material bangunan menawarkan solusi yang sangat potensial bagi pembangunan ramah lingkungan yang berkelanjutan. Dengan perkembangan teknologi dan kesadaran akan pentingnya keberlanjutan, kedua material ini memiliki peran yang sangat strategis dalam mendukung arsitektur yang lebih ramah lingkungan dan efisien. Penggunaan bambu dan kayu tidak hanya memberikan manfaat lingkungan, tetapi juga memberikan nilai tambah dari segi estetika dan kenyamanan bagi penghuni bangunan. bambu dan kayu terbukti sebagai material bangunan yang ramah lingkungan, berkelanjutan, dan relevan untuk mendukung konstruksi modern maupun tradisional di Indonesia. Keunggulan kedua material ini mencakup kekuatan struktural, kecepatan pertumbuhan (khususnya bambu), estetika alami, dan kemampuannya untuk diolah secara lokal. Dengan pendekatan desain yang tepat, bambu dan kayu dapat memberikan solusi arsitektur yang fungsional, estetis, dan berkontribusi pada pengurangan jejak karbon.

% subsection Kesimpulan (end)


% \bibliographystyle{apalike}
% \bibliography{../biblio.bib} % required for citecompletion, not work in mainfile
\end{document}

