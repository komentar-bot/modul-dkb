%! TEX root = elemenstruktur.tex
% the main TeX file which is intended to compile, :VimtexReload after adjustment
   % this file should be included in the main file
% :h vimtex-tex-root
\documentclass[../main.tex]{subfiles}
%ensure the class of docs

\begin{document}

\section{Struktur-Struktur Bangunan}%
\label{sec:Struktur-Struktur Bangunan}

\subsection{Pondasi}%
\label{sub:Pondasi}

\textbf{Pondasi} adalah elemen penting dalam konstruksi bangunan yang bertugas menyalurkan beban struktur ke tanah. Pondasi termasuk dalam sub struktur bangunan. Pondasi terbagi menjadi dua jenis utama, yaitu pondasi dangkal dan pondasi dalam, yang dipilih berdasarkan kondisi tanah, beban bangunan, dan jenis konstruksi. Pondasi dangkal digunakan jika beban bangunan dapat didistribusikan secara efektif pada lapisan tanah yang berada dekat dengan permukaan. Berikut jenis-jenisnya:

\begin{enumerate}
	\item Pondasi Telapak \textit{(Spread Footing)}: Pondasi ini berbentuk lempengan beton bertulang yang diletakkan langsung di bawah kolom atau dinding. Beban disebarkan ke area tanah yang lebih luas. Umumnya memiliki kedalaman 1,5-2 meter. Sangat cocok digunakan untuk bangunan kecil hingga menengah di tanah yang stabil.
	\item Pondasi Memanjang \textit{(Strip Footing)}: Pondasi ini berupa balok beton memanjang di bawah dinding atau struktur linear lainnya. Digunakan untuk menahan beban dari dinding. Umumnya digunakan pada bangunan sederhana seperti rumah tinggal.
	\item Pondasi Tikar \textit{(Raft Foundation/Mat Foundation)}: Pondasi ini menggunakan pelat beton bertulang besar yang mencakup seluruh area bangunan. Beban bangunan didistribusikan merata ke tanah. Seringkali digunakan pada tanah dengan daya dukung rendah atau untuk bangunan yang memiliki banyak kolom dan beban besar.
	\item Pondasi Cincin: Pondasi ini adalah salah satu jenis pondasi dangkal berbentuk lingkaran yang biasanya digunakan untuk menopang struktur silindris atau melingkar, seperti menara air, cerobong asap, silo, atau bangunan dengan kolom utama berbentuk bulat. Pondasi ini dirancang untuk mendistribusikan beban ke tanah secara merata melalui bentuk lingkarannya, sehingga memberikan stabilitas pada struktur. Materialnya terbuat dari beton bertulang agar mampu menahan beban vertikal maupun lateral. Termasuk pondasi dangkal karena biasanya berada dekat permukaan tanah, tetapi dapat disesuaikan berdasarkan analisis geoteknik.
\end{enumerate}

Pondasi Dalam digunakan untuk menyalurkan beban bangunan ke lapisan tanah yang lebih dalam, yang memiliki daya dukung lebih baik. Berikut adalah jenis pondasi dalam yang umum digunakan di Indonesia:

\tabitems Pondasi Tiang Pancang \textit{(Pile Foundation)}, pondasi dalam yang menggunakan elemen panjang berbentuk silindris yang ditanam ke dalam tanah untuk menyalurkan beban struktur ke lapisan tanah yang lebih stabil. Tiang pancang digunakan ketika lapisan tanah permukaan tidak memiliki daya dukung yang cukup untuk menopang bangunan. Umumnya digunakan untuk bangunan tinggi, jembatan, atau struktur di atas tanah lunak. Pondasi ini mengurangi risiko penurunan diferensial pada bangunan besar atau berat. Beberapa jenis tiang pancang dapat dilihat pada tabel~\ref{tb:jenistiang} dan Kelebihan vs kelemahan pondasi tiang pancang dapat dilihat pada tabel~\ref{tb:tiangpancang}

\begin{table}[htb]
	\caption{Jenis-jenis tiang pancang}
	\label{tb:jenistiang}
	\begin{tabularx}{0.8\textwidth}{|X|X|X|}
		\hline
		\textbf{Jenis Tiang Pancang}  &  \textbf{Deskripsi}                                                                                            &  \textbf{Kegunaan}  \\
		\hline
		Tiang beton                   &  Tiang pancang terbuat dari beton pracetak atau beton bertulang yang kuat menahan beban vertikal dan lateral.  &  Digunakan untuk bangunan besar seperti gedung tinggi dan jembatan.  \\
		\hline
		Tiang baja                    &  Tiang pancang terbuat dari beton pracetak atau beton bertulang yang kuat menahan beban vertikal dan lateral.  &  Cocok untuk tanah berbatu atau proyek berat seperti dermaga.  \\
		\hline
		Tiang kayu                    &  Tiang pancang tradisional yang terbuat dari batang kayu panjang.                                              &  Digunakan untuk konstruksi ringan di tanah basah atau rawa.  \\
		\hline
		Tiang komposit                &  Kombinasi dua material, seperti beton dan baja, untuk mendapatkan kekuatan dan fleksibilitas lebih baik.      &  Digunakan untuk proyek yang memerlukan daya tahan tinggi.  \\
		\hline
	\end{tabularx}
\end{table}

\citep{debatarajas2021}

\begin{table}[htb]
	\caption{Kelebihan vs Kelemahan Pondasi Tiang Pancang}
	\label{tb:tiangpancang}
	\begin{tabularx}{.8\textwidth}{lll}
		\hline
		Aspek            &  Kelebihan                                       &  Kekurangan  \\
		\hline
		Efesiensi beban  &  Dapat menahan beban besar                       &  Biaya tinggi untuk alat dan instalasi  \\
		Fleksibilitas    &  Cocok untuk berbagai kondisi tanah              &  Membutuhkan ruang kerja besar untuk alat berat  \\
		Stabilitas       &  Memberikan stabilitas terhadap gaya horizontal  &  Instalasi dapat menimbulkan getaran di sekitar area  \\
		Kecepatan        &  Cepat untuk proyek besar                        &  Tidak cocok untuk lokasi sempit atau padat.  \\
		\hline
	\end{tabularx}
\end{table}

\citep{supriyanto2019}

\tabitems Pondasi Bore Pile, salah satu jenis pondasi dalam yang
menggunakan metode pengeboran tanah untuk membuat
lubang di kedalaman tertentu, yang kemudian diisi dengan
beton bertulang. Proses ini dilakukan di lokasi proyek. Bore pile
banyak digunakan pada proyek-proyek konstruksi besar,
terutama di kawasan perkotaan, karena metode ini lebih minim
getaran dibandingkan dengan tiang pancang. Selain itu pondasi
bore pile mentransfer beban bangunan ke lapisan tanah keras di
kedalaman tertentu. Proses pengerjaan bore pile dapat dilihat
pada penjelasan gambar~\ref{fig:borepile} berikut:

\begin{figure}
	\begin{center}
		\includegraphics[width=.82\textwidth]{bore.jpeg}
	\end{center}
	\caption{Proses Pengerjaan Bore Pile}
	\label{fig:borepile}
\end{figure}

Di Indonesia kita mengenal ada 3 (tiga) jenis bore pile yaitu bore pile kering, Bore pile basah, dan Mini Pile.

No
Aspek
Bore Pile
Tiang Pancang

Metode Instalasi
Pengeboran dan pengecoran di lokasi.
Pemancangan tiang pracetak.

Getaran
Tidak ada getaran.
Menimbulkan getaran selama instalasi.

Lingkungan
Lebih ramah untuk area perkotaan.
Dapat mengganggu bangunan sekitar.

Kecepatan
Lebih lambat karena proses pengeboran.
Lebih cepat untuk proyek besar.

Kondisi
Tanah
Cocok untuk tanah
lunak.
Cocok untuk tanah
keras di kedalaman







\bibliographystyle{apalike}
\bibliography{../biblio.bib} % required for citecompletion, not work in mainfile
\end{document}

