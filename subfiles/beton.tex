%! TEX root = beton.tex
% the main TeX file which is intended to compile, :VimtexReload after adjustment
   % this file should be included in the main file
% :h vimtex-tex-root
\documentclass[../dkb.tex]{subfiles}
%ensure the class of docs

\begin{document}

\begin{comment}
\section{Struktur Beton}%
\label{sec:Struktur Beton}

Menurut SNI 03-2847-2002, beton didefinisikan sebagai "campuran antara semen Portland atau semen hidrolik lainnya, agregat halus, agregat kasar dan air, dengan atau tanpa bahan tambahan yang membentuk massa padat". Parameter penting dalam penentuan kualitas beton meliputi ± 15 \% semen, ± 8 \% air, ± 3 \% udara, selebihnya pasir dan kerikil. Beton akan mencapai kekuatan rencana (f’c) pada usia 28 hari Beton memliki daya kuat tekan yang baik.

Karakteristik beton mencakup beberapa aspek penting yang menjadikannya material konstruksi yang sangat efektif, yaitu:

\begin{enumerate}
	\item Beton memiliki kekuatan tekan yang sangat tinggi, sehingga sangat cocok digunakan untuk struktur yang menerima beban tekan, seperti kolom, dinding, dan pondasi.
	\item Beton juga memiliki kelemahan dalam tarik, karena material ini lemah terhadap gaya tarik; oleh karena itu, pada elemen yang mengalami gaya tarik atau lentur, biasanya ditambahkan baja tulangan untuk meningkatkan kekuatan tariknya.
	\item Beton memiliki daya tahan yang baik terhadap kondisi lingkungan yang ekstrem, seperti korosi, api, air, dan abrasi, menjadikannya ideal untuk struktur eksternal yang diharapkan memiliki umur panjang.
	\item Beton menawarkan kemudahan pengecoran, karena dapat dibentuk ke dalam berbagai bentuk dan ukuran menggunakan cetakan (bekisting), sehingga memberikan fleksibilitas dalam desain struktur.
\end{enumerate}

Terkadang, satu atau lebih bahan aditif ditambahkan ke dalam campuran beton untuk menghasilkan karakteristik tertentu, seperti kemudahan pengerjaan (workability), durabilitas, dan waktu pengerasan (McCormac, 2004). Beberapa jenis aditif yang umum digunakan antara lain, yaitu :
\begin{enumerate}
	\item Fly ash yaitu partikel butiran halus yang dihasilkan dari residu pembakaran batu bara.
	\item Slag merupakan material non-metal berbentuk halus yang dihasilkan dari proses pembakaran dan kemudian didinginkan dengan pencelupan ke dalam air.
	\item silica fume, yaitu material halus sisa dari produksi besi silikon yang mengandung banyak silica.
	\item dry dust collector, yang merupakan material halus yang berasal dari proses peleburan baja pada tahap akhir.
\end{enumerate}

Struktur beton adalah sistem konstruksi yang memanfaatkan beton sebagai bahan utama untuk menahan beban. Penggunaan struktur beton sangat umum dalam pembangunan gedung, jembatan, jalan, dan infrastruktur lainnya, karena beton memiliki kelebihan seperti kekuatan tekan yang tinggi dan ketahanan terhadap kondisi lingkungan yang ekstrem. Dengan karakteristik ini, struktur beton mampu memberikan stabilitas dan daya tahan yang diperlukan untuk berbagai jenis proyek konstruksi. Merancang bangunan dengan menggunakan struktur beton adalah pilihan yang populer karena sifatnya yang kuat, tahan lama, dan serbaguna. Desain struktur beton yang tepat sangat penting untuk memastikan keamanan dan ketahanan bangunan dalam jangka panjang. Proses perencanaan struktur beton melibatkan berbagai tahap yang saling terkait, mulai dari pemilihan material hingga analisis kebutuhan tulangan. Berikut adalah jenis-jenis beton


\end{comment}
\begin{table}[htb]
	\caption{Jenis-jenis beton}
	\label{tb:jenbeton}
	\begin{tabularx}{.93\textwidth}{p{2cm}Xp{3cm}X}
		Jenis beton                                                                                                                                                                   &  Deskripsi  &  Karakteristik  &  Penggunaan umum  \\
		Beton Bertulang (Reinforced Concrete)                                                                                                                                         &
		Beton yang diperkuat dengan tulangan baja untuk menahan gaya tarik dan lentur. Tulangan baja memberikan kekuatan pada beton dalam menahan beban tarik.                        &
		\tabitems Diperkuat dengan baja tulangan.
		\tabitems Dapat menahan beban tarik dan lentur.
		\tabitems Cocok untuk struktur besar dan kompleks.                                                                                                                            &
		Bangunan gedung bertingkat, jembatan, terowongna, pondasi dan lantai yang memerlukan ketahanan lentur dan tarik  \\
		Beton Pratekan (Prestressed Concrete)                                                                                                                                         &
		Beton yang diberi tegangan tarik pada baja tulangannya sebelum proses pencorannya, sehingga setelah beton mengeras, tegangan tersebut dilepaskan dan menciptakan gaya tekan.  &
		\tabitems Baja tulangan diberikan tegangan sebelum pencoran.
		\tabitems Meningkatkan kapasitas lentur beton.
		\tabitems Mengurangi retak pada beton.                                                                                                                                        &
		Jembatan dengan bentang panjang, lantai parkir bertingkat, struktur yang membutuhka n kekuatan lentur tinggi dan ketahanan terhadap beban dinamis  \\

		Beton Pracetak (Precast Concrete)                                                                                                                                             &
		Elemen beton yang dicor di pabrik dan kemudian dipasang di lokasi konstruksi. Metode ini mempercepat proses konstruksi dan memastikan kualitas lebih baik pada elemen beton.  &
		\tabitems Diproduksi di pabrik dan dipasang di lokasi.
		\tabitems Kualitas lebih terjamin karena proses produksi di bawah kendali ketat.
		\tabitems Mudah untuk dipasang dan dapat diproduksi dalam jumlah besar.                                                                                                       &
		Panel dinding dan lantai gedung bertingkat, struktur jembatan dan saluran, sistem drainase dan saluran air.  \\
	\end{tabularx}
\end{table}

\subsection{Pemilihan Material Beton}%
\label{sub:Pemilihan Material Beton}

Pemilihan material beton yang tepat merupakan langkah awal yang sangat penting dalam proses konstruksi. Dalam menentukan material, faktor-faktor seperti kekuatan tekan, ketahanan terhadap air, dan ketahanan terhadap suhu ekstrem harus dipertimbangkan dengan cermat. Jenis semen, agregat kasar dan halus, serta air yang digunakan akan sangat mempengaruhi sifat akhir beton. Beton memiliki sejumlah keunggulan yang membuatnya menjadi pilihan utama dalam konstruksi, yaitu:

\begin{enumerate}
	\item Kemudahan Pengerjaan (workability) membuat beton mudah dicetak dalam berbagai bentuk dan ukuran sesuai kebutuhan konstruksi.
	\item Kekuatan Tekan yang cukup tinggi, dan ketika dikombinasikan dengan baja tulangan yang memiliki kekuatan tarik tinggi, keduanya dapat membentuk satu kesatuan struktur yang tahan terhadap gaya tarik dan tekan, sehingga mampu memikul beban berat.
	\item tahan terhadap temperatur tinggi dan merupakan material yang awet, tahan aus, serta tahan terhadap panas dan pengkaratan atau pembusukan akibat kondisi lingkungan.
	\item harga yang relatif lebih murah karena menggunakan bahanbahan dasar yang umumnya mudah didapat, serta biaya pemeliharaan yang kecil. Dengan berbagai keunggulan ini, beton menjadi material yang sangat efektif untuk berbagai proyek konstruksi.
\end{enumerate}

Material penyusun beton terdiri dari beberapa komponen utama yang bekerja secara sinergis untuk membentuk struktur beton yang kuat dan tahan lama. Material utama dalam campuran beton meliputi semen, agregat kasar dan halus, air, serta bahan tambahan (admixture) yang dapat meningkatkan sifat beton sesuai kebutuhan. Kombinasi dari material-material ini harus diperhitungkan dengan baik agar beton memiliki kualitas yang optimal sesuai standar konstruksi.


\subsubsection{Semen}%
\label{ssub:Semen}

Semen adalah bahan utama dalam beton yang berfungsi sebagai perekat antara agregat kasar dan halus. Jenis semen yang digunakan dalam konstruksi beton sangat beragam, tergantung pada kebutuhan proyek. Beberapa jenis semen yang umum digunakan dapat dilihat pada gambar~\ref{fig:jensemen}. Semen Portland (OPC-Ordinary Portland Cement) yaitu jenis semen paling umum yang digunakan untuk berbsgai konstruksi bangunan. Selain itu ada PPC (Portland Pozzolan Cement) yang mengandung bahan pozzolan yang meningkatkan ketahanan terhadap sulfat dan mengurangi panas hidrasi.

\begin{figure}
	\begin{center}
		\includegraphics[width=.82\textwidth]{../figures/jenis_semen.jpeg}
		% filename may not contain space but _
	\end{center}
	\caption{Jenis-jenis semen}
	\label{fig:jensemen}
\end{figure}

Semen memiliki beberapa sifat utama yang mempengaruhi kinerja beton, diantaranya Fineness (Kehalusan) yang berarti semakin halus semen, semakin cepat reaksi hoidrasi yang terjadi.




\subsubsection{Agregat Kasar dan Ageregat Halus}%
\label{ssub:Agregat Kasar dan Ageregat Halus}

Agregat kasar berfungsi sebagai bahan pengisi utama dalam beton yang memberikan kekuatan struktural. Yang termasuk Agregat kasar adalah kerikil, batu pecah, dan slag (limbah industri). Sedangkan Agregat halus berperan dalam mengisi celah antar agregat kasar dan membantu meningkatkan workability beton. Yang termasuk agregat halus adalah pasir sungai, pasir buatan, dan pasir laut.

\subsubsection{Air}%
\label{ssub:Air}
Air memiliki fungsi utama dalam proses hidrasi semen, yang merupakan reaksi kimia antara semen dan air untuk membentuk pasta semen yang mengikat agregat. Jumlah air yang digunakan dalam campuran beton sangat mempengaruhi kekuatan dan daya tahan beton. Terlalu banyak air maka mengakibatkan beton menjadi lemah, mudah retak, dan berpori. Namun terlalu sedikit air maka mengakibatkan campuran beton sulit diolah dan tidak dapat mengisi cetakan dengan baik.

\subsubsection{Bahan tambahan (Admixture) dalam Beton}%
\label{ssub:Bahan tambahan (Admixture) dalam Beton}

Adalah zat yang ditambahkan ke dalam campuran beton untuk meningkatkan sifat tertentu, baik dalam hal kekuatan, durabilitas, maupun kemudahan pengerjaan. Salah satu jenis admixture yang umum digunakan adalah superplasticizer, yang berfungsi meningkatkan workability tanpa menambah jumlah air dalam campuran beton, sehingga cocok untuk beton dengan kepadatan tinggi dan kebutuhan pengecoran yang lebih mudah. Selain itu, terdapat retarder, yang berguna untuk memperlambat proses pengerasan beton, terutama dalam kondisi cuaca panas atau proyek dengan volume pengecoran besar yang membutuhkan waktu lebih lama sebelum beton mulai mengeras. Di sisi lain, accelerator digunakan untuk mempercepat proses pengerasan beton, sehingga sangat ideal bagi proyek yang memerlukan konstruksi cepat, seperti perbaikan jalan atau struktur yang membutuhkan kekuatan awal tinggi. Sementara itu, airentraining agent berfungsi untuk membentuk gelembung udara kecil dalam beton, yang membantu meningkatkan ketahanan terhadap siklus pembekuan dan pencairan, serta meningkatkan daya tahan beton terhadap lingkungan ekstrem. Untuk meningkatkan ketahanan terhadap air, waterproofing admixture ditambahkan ke dalam campuran beton guna mengurangi permeabilitasnya. Admixture ini sangat cocok digunakan pada konstruksi bawah tanah, seperti basement, terowongan, atau bangunan yang membutuhkan ketahanan tinggi terhadap kelembapan. Dengan penggunaan bahan tambahan yang tepat, kualitas beton dapat ditingkatkan sesuai dengan kebutuhan spesifik dalam setiap proyek konstruksi.

\subsection{Prinsip Dasar Desain Struktur Beton}%
\label{sub:Prinsip Dasar Desain Struktur Beton}

Desain struktur beton didasarkan pada prinsip-prinsip dasar yang mengutamakan keamanan, kenyamanan, dan efisiensi biaya. Beberapa prinsip utama dalam desain struktur beton meliputi kekuatan dan ketahanan, Stabilitas dan kekakuan, Durabilitas, dan Keamanan.

\subsubsection{Perancangan beton bertulang}%
\label{ssub:Perancangan beton bertulang}
\tabitems Desain Balok

Balok beton bertulang dirancang untuk menahan beban lentur yang terjadi akibat distribusi beban dari plat lantai atau elemen struktur lainnya. Proses perancangannya melibatkan:

\begin{enumerate}
	\item Penentuan ebban dan Momen Lentur: Beban yang bekerja pada balok dihitung berdasarkan kondisi beban mati dan beban hidup
	\item Pengaturan dimensi balok: Dimensi balok (tinggi, lebar) disesuaikan dengan kapasitas lentur yang dibutuhkan untuk menahan beban.
	\item Penempatan tulangan: Baja tulangan diletakkan pada bagian bawah balok untuk menahan gaya tarik dan bagian atas untuk menahan gaya tekan.
\end{enumerate}

\tabitems Desain Kolom

Kolom adalah elemen vertikal yang menahan beban tekan. Desain kolom harus memastikan bahwa kolom dapat menahan beban vertikal yang besar tanpa mengalami keruntuhan atau deformasi yang berlebihan.

\begin{enumerate}
	\item Perhitungan beban tekan: Beban yang diterima oleh kolom dihitung berdasarkan jumlah beban dari lantailantai di atasnya.
	\item Dimensi kolom: Dimensi kolom ditentukan berdasarkan perhitungan kekuatan tekan dan ketahanan terhadap beban yang bekerja.
	\item Penulangan kolom: Kolom dirancang dengan penulangan yang kuat di seluruh bagian kolom untuk menahan tegangan tekan.
\end{enumerate}

\tabitems Desain plat dan slab

Plat dan slab adalah elemen datar yang menahan beban dari lantai atau atap. Dalam desainnya, penting untuk memperhitungkan kapasitas lentur dan distribusi beban yang merata.

\begin{enumerate}
	\item Desain plat bertulang: Plat dengan tulangan harus dirancang untuk menahan beban yang bekerja di atasnya, dengan mempertimbangkan pembagian beban pada panjang dan lebar plat.
	\item Penggunaan tulangan datar dan vertikal: Desain tulangan untuk plat dilakukan dengan menempatkan tulangan utama secara horizontal untuk menahan lentur dan tulangan bantu secara vertikal untuk mengurangi kemungkinan keretakan.
\end{enumerate}







\subsubsection{Perancangan Beton Pratekan}%
\label{ssub:Perancangan Beton Pratekan}

\begin{enumerate}
	\item  Proses Pratekan
	\item[] Proses pratekan dilakukan dengan menegangkan baja tulangan (untuk beton pratekan) sebelum beton dicor. Setelah beton mengeras, tegangan tersebut dilepaskan, sehingga beton mengalami gaya tekan yang meningkatkan kapasitas lenturnya.
	\item Desain Struktur beton pratekan
	\item[] Desain struktur beton pratekan melibatkan analisis terhadap gaya tarik dan tekan yang dihasilkan pada baja tulangan, serta pengaruhnya terhadap beton.
\end{enumerate}

\subsection{Pengaruh Beban pada Struktur Beton}%
\label{sub:Pengaruh Beban pada Struktur Beton}

Beban yang bekerja pada struktur beton sangat mempengaruhi kinerja dan desain struktur tersebut. Beban yang diterima oleh beton dapat menyebabkan tegangan, deformasi, dan bahkan kerusakan jika tidak dihitung dan diperhitungkan dengan tepat. Oleh karena itu, penting untuk mengklasifikasikan jenis beban berdasarkan karakteristiknya, untuk mempermudah perhitungan dan perancangan yang lebih akurat dan aman.

\begin{enumerate}
	\item Beban Statis
	\item[] Beban statis adalah beban yang bersifat konstan dan tidak berubah seiring waktu. Beban ini biasanya berasal dari bagian struktur itu sendiri, yang dikenal dengan istilah beban mati. Contohnya adalah: Berat beton, baja, dan material lain yang membentuk struktur. Beban statis ini dihitung berdasarkan berat material yang ada pada struktur. Beton yang digunakan untuk menahan beban statis harus didesain untuk mengatasi tekanan dari beban mati yang besar.
	\item Beban Dinamis
	\item[] Beban dinamis adalah beban yang berubah-ubah seiring waktu, dan dapat memengaruhi struktur dengan cara yang lebih kompleks. Beban ini menyebabkan fluktuasi dalam tekanan dan momen yang bekerja pada struktur. Contoh: Beban kendaraan yang melintasi jembatan atau struktur parkir. Beban dinamis harus dihitung dengan lebih cermat, karena dapat menghasilkan getaran atau perubahan tegangan yang lebih besar dibandingkan dengan beban statis.
	\item Beban Hidup
	\item[] Beban hidup adalah beban yang berubah sesuai dengan penggunaan ruang atau aktivitas yang terjadi di dalam atau di sekitar bangunan. Beban hidup ini dapat bervariasi dari waktu ke waktu, tergantung pada apa yang ada di dalam bangunan dan berapa banyak orang yang berada di dalamnya. Contoh dari beban hidup adalah: Beban yang berasal dari orang-orang yang menggunakan ruang tersebut. Beban hidup ini sering kali lebih sulit diprediksi secara tepat karena sangat tergantung pada faktor-faktor yang bervariasi dalam penggunaan bangunan. Namun, perhitungan beban hidup ini penting untuk memastikan struktur beton dapat menahan beban yang fluktuatif ini dengan aman.
	\item Beban Mati
	\item[] Beban mati adalah beban yang bersifat tetap dan selalu ada selama masa penggunaan bangunan. Beban ini berasal dari berat struktur itu sendiri dan material bangunan yang digunakan. Contoh dari beban mati adalah: Berat beton, balok, kolom, dan elemen struktural lainnya. Beban mati biasanya lebih mudah dihitung karena sifatnya yang tetap dan tidak berubah seiring waktu. Namun, penting untuk memperhitungkan beban mati secara akurat untuk memastikan bahwa struktur beton memiliki kapasitas yang cukup untuk menahan beban tersebut tanpa mengalami kegagalan.
\end{enumerate}

\subsection{Metode Analisis Struktur Beton}%
\label{sub:Metode Analisis Struktur Beton}


Untuk memastikan bahwa struktur beton dapat bekerja secara aman dan efisien, berbagai metode analisis digunakan untuk memeriksa pengaruh beban dan memastikan struktur dapat menahan gaya yang bekerja dengan baik. Beberapa metode analisis struktur beton yang umum digunakan dapat dilihat pada gambar berikut:

\begin{figure}
	\begin{center}
		\includegraphics[width=.82\textwidth]{../figures/analisis_beton.jpeg}
		% filename may not contain space but _
	\end{center}
	\caption{Metode Analaisis Struktur Beton}
	\label{fig:anabeton}
\end{figure}

Inovasi dalam teknologi beton terus berkembang untuk memenuhi kebutuhan konstruksi modern yang mengutamakan efisiensi, keberlanjutan, dan keamanan. Beberapa teknologi mutakhir yang menarik perhatian dalam dunia konstruksi meliputi beton dengan fungsi khusus dan estetika yang unik. Inovasi dalam teknologi beton memberikan solusi baru untuk tantangan yang dihadapi dalam dunia konstruksi modern. Dengan mengintegrasikan teknologi mutakhir dan pendekatan ramah lingkungan, struktur beton masa depan tidak hanya lebih kuat dan tahan lama, tetapi juga lebih berkelanjutan. Arah pengembangan ini mencerminkan komitmen industri konstruksi untuk menghadirkan solusi yang relevan dengan kebutuhan masyarakat global di masa depan.

\subsection{Perbaikan dan Pemeliharaan Struktur Beton}%
\label{sub:Perbaikan dan Pemeliharaan Struktur Beton}


Kerusakan pada struktur beton adalah salah satu masalah utama yang dapat memengaruhi kekuatan, ketahanan, dan umur panjang bangunan. Penting untuk memahami penyebab dan jenis kerusakan yang terjadi agar dapat diambil langkah-langkah yang tepat dalam pemeliharaan dan perbaikan. Menurut (Rofiq, 2020) Beberapa penyebab kerusakan beton

\begin{description}
	\item[Faktor lingkungan:] Paparan air, kelembapan, atau zat kimia seperti karbonasi, klorida, atau sulfat dapat merusak beton dan tulangan di dalamnya.
	\item[Kondisi Mekanik:] Beban berlebih, getaran, atau gaya dinamis yang tidak sesuai desain dapat menyebabkan retak dan deformasi.
	\item[Kesalahan konstruksi:] Campuran beton yang tidak tepat, pencampuran yang buruk, atau pemasangan tulangan yang salah dapat mengurangi kualitas beton.
	\item[Kelebihan beban dan usia struktur:] Struktur yang terlalu tua atau menerima beban melebihi kapasitas desainnya lebih rentan terhadap kerusakan.
\end{description}

Jenis-jenis kerusakan beton dapat dilihat pada gambar~\ref{fig:rusakbeton}. dibawah ini. Keursakan permukaan termasuyk retak halus (hairline cracks), keropos (honeycombing), atau pengelupasan (spalling) merupakan kerusakan yang paling sering terjadi pada beton.

\begin{figure}
	\begin{center}
		\includegraphics[width=.82\textwidth]{../figures/rusak_beton.jpeg}
		% filename may not contain space but _
	\end{center}
	\caption{Jenis Kerusakan pada Beton}
	\label{fig:rusakbeton}
\end{figure}

Pemeliharaan dan perawatan struktur beton bertujuan untuk mencegah kerusakan lebih lanjut dan memastikan bahwa struktur tetap berfungsi dengan baik selama masa pakainya. Langkahlangkah utama meliputi:
\begin{enumerate}
	\item Inspeksi berkala: Pemeriksaan visual dan pengujian nondestruktif untuk mendeteksi retak, deformasi, atau kerusakan lainnya pada tahap awal.
	\item Perlindungan beton: Menggunakan pelapis tahan air atau bahan kimia pelindung untuk mencegah penetrasi air atau zat agresif ke dalam beton.
	\item Pembersihan dan perawatan rutin: Membersihkan permukaan beton dari debu, kotoran, atau zat kimia yang dapat menyebabkan kerusakan.
	\item
\end{enumerate}
Perbaikan struktur beton melibatkan berbagai metode dan teknologi untuk mengembalikan kekuatan serta fungsi struktur yang telah mengalami kerusakan.












\bibliographystyle{apalike}
\bibliography{../biblio.bib} % required for citecompletion, not work in mainfile
\end{document}

