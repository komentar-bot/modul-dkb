%! TEX root = definisikonstruksi.tex
% the main TeX file which is intended to compile, :VimtexReload after adjustment
   % this file should be included in the main file
% :h vimtex-tex-root
\documentclass[../dkb.tex]{subfiles}
%ensure the class of docs

\begin{document}

\section{Struktur Bangunan}%

Seni bangunan atau arsitektur adalah seni sejak adanya manusia dan disebut seni terikat, karena bangunan gedung dipakai oleh manusia dan bahan-bahan bangunan yang sifatnya dibatasi kemampuannya. Seni bangunan adalah seni dan teknik dengan mengikutsertakan faktor-faktor falsafah, religi, tradisi, seni dan ilmu pengetahuan.

Struktur bangunan adalah komponen penting dalam arsitektur. Tidak ada bedanya apakah bangunan dengan strukturnya hanya tempat untuk berlindung satu keluarga yang bersifat sederhana, ataukah tempat berkumpul atau bekerja bagi banyak orang, seperti perkantoran, gedung ibadah, hotel, gedung bioskop, stasiun dan sebagainya. Maka fungsi dari struktur ialah untuk melindungi suatu ruang tertentu terhadap iklim, bahaya-bahaya yang ditimbulkan alam dan menyalurkannya semua macam beban ke tanah. Beban-beban yang dipikulnya, berat bahan dari elemen-elemen beserta berat strukturnya sendiri disalurkan oleh struktur atau kerangka bangunan kekulit bumi. Kecuali beban tersebut, struktur harus dapt memikul beban lain akibat dari angin dan gempa bumi.

\textit{Struktur Bangunan Gedung adalah oganisasi daripada elemen-elemen ataupun komponen-komponen bangunan yang mendukung dapat berfungsinya bangunan gedung dengan baik}. Sistem struktur adalah bentuk organisasi daripada elemen-elemen struktur yang ditujukan untuk menyalurkan beban secara karakteristik.

Struktur adalah bagian-bagian yang membentuk bangunan seperti pondasi, sloof, dinding, kolom, ring, kuda-kuda, dan atap. Pada prinsipnya, elemen struktur berfungsi untuk mendukung keberadaan elemen nonstruktur yang meliputi elemen tampak, interior, dan detail arsitektur sehingga membentuk satu kesatuan. Setiap bagian struktur bangunan tersebut juga mempunyai fungsi dan peranannya masing-masing.

Kegunaan lain dari struktur bangunan yaitu meneruskan beban bangunan dari bagian bangunan atas menuju bagian bangunan bawah, lalu menyebarkannya ke tanah. Perancangan struktur harus memastikan bahwa bagian-bagian sistem struktur ini sanggup mengizinkan atau menanggung gaya gravitasi dan beban bangunan, kemudian menyokong dan menyalurkannya ke tanah dengan aman.

Pembebanan struktur bangunan adalah beraneka ragam dan rumit \textbf{(kompleks)}. Bangunan menampung orang-orang yang hidup, barang-barang yang dapat dipindahkan, beban angin yang berubah-ubah, berat struktur dan bahan-bahan bangunan yang statis semuanya dipikul ileh struktur atau kerangka bangunan dan sisalurkan ketanah melalui pondasi.

\subsection{Jenis Penyaluran Beban Struktur}%

\begin{description}
	\item[Struktur Utama] adalah organisasi dari elemen-elemen ataupun komponen- komponen bangunan yang menyalurkan beban ketanah dan tanpa adanya struktur ini bangunan tidak dapat berfungsi dengan baik.
	\item[Struktur Pendukung] adalah susunan elemen-elemen ataupun komponen bangunan yang mendukung struktur utama supaya dapat melaksanakan fungsinya dengan baik.
\end{description}

Banyak variasi pembebanan pada struktur bangunan. Beban-beban tersebut diatas dapat ditentukan dan diberi kode atau tanda dalam perencanaan struktur. Beban dibedakan menjadi:

\begin{enumerate}
	\item Beban Mati
	\item Beban Hidup
	\item Beban Angin
	\item Beban Termis
	\item Gerakan bangunan akibat gerakan tanah
	\item Goyangan bangunan akibat gempa bumi
	\item Beban Dinamis
\end{enumerate}

\subsection{Peran Struktur Bangunan}%

\begin{enumerate}
	\item Estetika, sebagai dasar keindahan dan keserasian bangunan yang mampu memberikan rasa bangga kepada pemiliknya.
	\item Fungsional, disesuaikan dengan pemanfaatan dan penggunaannya sehingga dalam pemakaiannya dapat memberikan kenikmatan dan kenyamanan.
	\item Struktural, mempunyai struktur yang kuat dan mantap yang dapat memberikan kenikmatan dan kenyamanan.
	\item Ekonomis, pendimensian elemen bangunan yang proporsional dan penggunaan bahan bangunan yang memadai sehingga bangunan awet dan mempunyai umur pakai yang panjang.
\end{enumerate}

\begin{table}[htb]
	\caption{Perkembangan Sejarah Perancangan Bangunan}
	\centering
    \begin{tabularx}{0.89\textwidth}{c p{2cm} X X}
		\hline
		No.  &  Peradaban                                                                                                    &  Bangunan Ikonik  &  Fungsi  \\
		\hline
		1    &  Prasejarah
		     &  Gua, gubuk
		     &  Perlindungan dari cuaca dan binatang buas.  \\

		2    &  Peradaban Kuno
		     &  Piramida Mesir; Ziggurat (Mesopotamia); Parthenon (Yunani); Colosseum (Romawi)
		     &  Fungsi religius dan monumental; perkembangan teknik konstruksi seperti kolom, lengkungan, kubah, dan beton.  \\

		3    &  Abad Pertengahan
		     &  Gereja dan kastil; Katedral Gotik (misalnya Notre Dame)
		     &  Bangunan pertahanan dan keagamaan dengan dinding tebal, lengkungan runcing, dan jendela kaca patri.  \\

		4    &  Abad Pencerahan dan Revolusi Industri
		     &  Jembatan dan gedung pabrik
		     &  Bangunan mulai bersifat fungsional; penggunaan material besi dan baja dalam konstruksi.  \\

		5    &  Abad ke-20 (Modernisme)
		     &  Gedung bertingkat dan pencakar langit
		     &  Dominasi struktur baja dan kaca dengan pendekatan desain fungsional.  \\

		6    &  Abad ke-21 (Kontemporer)
		     &  Gedung pintar dan ramah lingkungan
		     &  Fokus pada keberlanjutan, efisiensi energi, dan penggunaan material inovatif serta daur ulang.  \\
		\hline
	\end{tabularx}
\end{table}


\subsection{Dasar Konstruksi Bangunan}%

Dasar-dasar konstruksi bangunan adalah prinsip-prinsip dan elemen-elemen fundamental yang menjadi panduan dalam merancang, membangun, dan memelihara struktur bangunan. Pemahaman tentang dasar-dasar ini sangat penting bagi semua yang terlibat dalam industri konstruksi, seperti insinyur, arsitek, dan kontraktor. Dasar-dasar konstruksi melibatkan berbagai aspek mulai dari material, metode, hingga pengelolaan proyek. Seperti dijelaskan pada bab sebelumnya prinsip utama dalam konstruksi bangunan haruslah mampu dalam kekuatan dan stabilitas yang berarti struktur harus mampu menahan beban vertikal (beban mati dan beban hidup) serta beban horizontal (angin, gempa). Selain itu efisiensi material, keberlanjutan, dan aspek keamanan juga harus menjadi prinsip utama.

Sistem struktur dalam hubungannya dengan bangunan ialah bahwa struktur merupakan sarana untuk menyalurkan beban akibat penggunaan dan atau kehadiran bangunan ke dalam tanah. Sistem strukutr pada bangunan berlantai dapat diklasifikasikan menjadi 3 (tiga) yaitu:

\begin{enumerate}
	\item Struktur bawah \textit{(sub structure)} adalah bagian-bagian bangunan yang terletak di bawah permukaan tanah. Contohnya pondasi, fungsi pondasi sebagai penerima gaya yang akan disalurkan ke tanah.
	\item Struktur tengah \textit{(middle structure)} merupakan bagian-bagian bangunan yang terletak di atas permukaan tanah dan di bawah atap. Contohnya struktur berupa kolom, balok, plat lantai. Bagian ini berada pada bagian badan bangunan yang mana fungsinya sebagai penyalur gaya di dalam bangunan.
	\item Struktur atas \textit{(up/super structure)} yaitu bagian-bagian bangunan yang terbentuk memanjang ke atas untuk menopang atap. Contohnya kuda-kuda yang berfungsi sebagai penopang material penutup. Kuda-kuda juga berguna sebagai penyalur beban dari atap.
\end{enumerate}

\subsubsection{Jenis-jenis beban yang diterima}%

\begin{enumerate}
	\item Beban Mati
	\item Beban Hidup
	\item Beban Angin
	\item Beban Gempa
	\item Beban Salju
	\item Beban Hujan
	\item Beban Termal
	\item Beban Ledakan
	\item Beban Tanah
	\item Beban Air
\end{enumerate}

\citep{supriyantob.2019}

Sistem konstruksi dalam bangunan merupakan bagian atau elemen yang menempel pada system struktur utama, sedangkan fungsi dari system konstruksi adalah elemen yang dapat menyebarkan gaya dan penerima beban secara langsung.

Penempatan sistem konstruksi pada bangunan berlantai berada pada:

\begin{description}
	\item[Super Struktur] berupa tangga, dinding, plafond. Fungsi system konstruksi yang beraada pada bagian super struktur adalah menyalurkan gaya-gaya ke system struktur bangunan.
	\item[Up Struktur] berupa atap, listplank, talang air. Fungsi system konstruksi yang berada pada bagian up struktur adalah penerima beban secara langsung. Beban yang diterima berupa beban angin dan hal ini terjadi pada system konstruksi atap, sedangkan listplank berfungsi sebagai penrima beban angin dari arah samping atap sedangkan talang air berfungsi sebagai penyalur air hujan pada atap dan talang air juga dapat berfungsi sebagai pembentuk atap.
\end{description}

% \bibliographystyle{apalike}
% \bibliography{../biblio.bib} % required for citecompletion, not work in mainfile

\end{document}

